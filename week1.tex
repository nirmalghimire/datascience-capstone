% Options for packages loaded elsewhere
\PassOptionsToPackage{unicode}{hyperref}
\PassOptionsToPackage{hyphens}{url}
%
\documentclass[
]{article}
\usepackage{lmodern}
\usepackage{amssymb,amsmath}
\usepackage{ifxetex,ifluatex}
\ifnum 0\ifxetex 1\fi\ifluatex 1\fi=0 % if pdftex
  \usepackage[T1]{fontenc}
  \usepackage[utf8]{inputenc}
  \usepackage{textcomp} % provide euro and other symbols
\else % if luatex or xetex
  \usepackage{unicode-math}
  \defaultfontfeatures{Scale=MatchLowercase}
  \defaultfontfeatures[\rmfamily]{Ligatures=TeX,Scale=1}
\fi
% Use upquote if available, for straight quotes in verbatim environments
\IfFileExists{upquote.sty}{\usepackage{upquote}}{}
\IfFileExists{microtype.sty}{% use microtype if available
  \usepackage[]{microtype}
  \UseMicrotypeSet[protrusion]{basicmath} % disable protrusion for tt fonts
}{}
\makeatletter
\@ifundefined{KOMAClassName}{% if non-KOMA class
  \IfFileExists{parskip.sty}{%
    \usepackage{parskip}
  }{% else
    \setlength{\parindent}{0pt}
    \setlength{\parskip}{6pt plus 2pt minus 1pt}}
}{% if KOMA class
  \KOMAoptions{parskip=half}}
\makeatother
\usepackage{xcolor}
\IfFileExists{xurl.sty}{\usepackage{xurl}}{} % add URL line breaks if available
\IfFileExists{bookmark.sty}{\usepackage{bookmark}}{\usepackage{hyperref}}
\hypersetup{
  pdftitle={week1 Capstone},
  pdfauthor={Nirmal Ghimire},
  hidelinks,
  pdfcreator={LaTeX via pandoc}}
\urlstyle{same} % disable monospaced font for URLs
\usepackage[margin=1in]{geometry}
\usepackage{color}
\usepackage{fancyvrb}
\newcommand{\VerbBar}{|}
\newcommand{\VERB}{\Verb[commandchars=\\\{\}]}
\DefineVerbatimEnvironment{Highlighting}{Verbatim}{commandchars=\\\{\}}
% Add ',fontsize=\small' for more characters per line
\usepackage{framed}
\definecolor{shadecolor}{RGB}{248,248,248}
\newenvironment{Shaded}{\begin{snugshade}}{\end{snugshade}}
\newcommand{\AlertTok}[1]{\textcolor[rgb]{0.94,0.16,0.16}{#1}}
\newcommand{\AnnotationTok}[1]{\textcolor[rgb]{0.56,0.35,0.01}{\textbf{\textit{#1}}}}
\newcommand{\AttributeTok}[1]{\textcolor[rgb]{0.77,0.63,0.00}{#1}}
\newcommand{\BaseNTok}[1]{\textcolor[rgb]{0.00,0.00,0.81}{#1}}
\newcommand{\BuiltInTok}[1]{#1}
\newcommand{\CharTok}[1]{\textcolor[rgb]{0.31,0.60,0.02}{#1}}
\newcommand{\CommentTok}[1]{\textcolor[rgb]{0.56,0.35,0.01}{\textit{#1}}}
\newcommand{\CommentVarTok}[1]{\textcolor[rgb]{0.56,0.35,0.01}{\textbf{\textit{#1}}}}
\newcommand{\ConstantTok}[1]{\textcolor[rgb]{0.00,0.00,0.00}{#1}}
\newcommand{\ControlFlowTok}[1]{\textcolor[rgb]{0.13,0.29,0.53}{\textbf{#1}}}
\newcommand{\DataTypeTok}[1]{\textcolor[rgb]{0.13,0.29,0.53}{#1}}
\newcommand{\DecValTok}[1]{\textcolor[rgb]{0.00,0.00,0.81}{#1}}
\newcommand{\DocumentationTok}[1]{\textcolor[rgb]{0.56,0.35,0.01}{\textbf{\textit{#1}}}}
\newcommand{\ErrorTok}[1]{\textcolor[rgb]{0.64,0.00,0.00}{\textbf{#1}}}
\newcommand{\ExtensionTok}[1]{#1}
\newcommand{\FloatTok}[1]{\textcolor[rgb]{0.00,0.00,0.81}{#1}}
\newcommand{\FunctionTok}[1]{\textcolor[rgb]{0.00,0.00,0.00}{#1}}
\newcommand{\ImportTok}[1]{#1}
\newcommand{\InformationTok}[1]{\textcolor[rgb]{0.56,0.35,0.01}{\textbf{\textit{#1}}}}
\newcommand{\KeywordTok}[1]{\textcolor[rgb]{0.13,0.29,0.53}{\textbf{#1}}}
\newcommand{\NormalTok}[1]{#1}
\newcommand{\OperatorTok}[1]{\textcolor[rgb]{0.81,0.36,0.00}{\textbf{#1}}}
\newcommand{\OtherTok}[1]{\textcolor[rgb]{0.56,0.35,0.01}{#1}}
\newcommand{\PreprocessorTok}[1]{\textcolor[rgb]{0.56,0.35,0.01}{\textit{#1}}}
\newcommand{\RegionMarkerTok}[1]{#1}
\newcommand{\SpecialCharTok}[1]{\textcolor[rgb]{0.00,0.00,0.00}{#1}}
\newcommand{\SpecialStringTok}[1]{\textcolor[rgb]{0.31,0.60,0.02}{#1}}
\newcommand{\StringTok}[1]{\textcolor[rgb]{0.31,0.60,0.02}{#1}}
\newcommand{\VariableTok}[1]{\textcolor[rgb]{0.00,0.00,0.00}{#1}}
\newcommand{\VerbatimStringTok}[1]{\textcolor[rgb]{0.31,0.60,0.02}{#1}}
\newcommand{\WarningTok}[1]{\textcolor[rgb]{0.56,0.35,0.01}{\textbf{\textit{#1}}}}
\usepackage{graphicx,grffile}
\makeatletter
\def\maxwidth{\ifdim\Gin@nat@width>\linewidth\linewidth\else\Gin@nat@width\fi}
\def\maxheight{\ifdim\Gin@nat@height>\textheight\textheight\else\Gin@nat@height\fi}
\makeatother
% Scale images if necessary, so that they will not overflow the page
% margins by default, and it is still possible to overwrite the defaults
% using explicit options in \includegraphics[width, height, ...]{}
\setkeys{Gin}{width=\maxwidth,height=\maxheight,keepaspectratio}
% Set default figure placement to htbp
\makeatletter
\def\fps@figure{htbp}
\makeatother
\setlength{\emergencystretch}{3em} % prevent overfull lines
\providecommand{\tightlist}{%
  \setlength{\itemsep}{0pt}\setlength{\parskip}{0pt}}
\setcounter{secnumdepth}{-\maxdimen} % remove section numbering

\title{week1 Capstone}
\author{Nirmal Ghimire}
\date{1/4/2021}

\begin{document}
\maketitle

\hypertarget{tasks-to-accomplish}{%
\subsection{Tasks to accomplish}\label{tasks-to-accomplish}}

\textbf{\emph{Tokenization }}- identifying appropriate tokens such as
words, punctuation, and numbers. Writing a function that takes a file as
input and returns a tokenized version of it.

\textbf{\emph{Profanity filtering}} - removing profanity and other words
you do not want to predict.

\hypertarget{tips-tricks-and-hints}{%
\subsubsection{Tips, Tricks, and Hints}\label{tips-tricks-and-hints}}

\textbf{\emph{Loading the data in}}.

\begin{enumerate}
\def\labelenumi{\alph{enumi}.}
\tightlist
\item
  This dataset is fairly large. We emphasize that you don't necessarily
  need to load the entire dataset in to build your algorithms (see point
  2 below).
\item
  At least initially, you might want to use a smaller subset of the
  data. Reading in chunks or lines using R's \textbf{readLines} or
  \textbf{scan} functions can be useful.
\item
  You can also loop over each line of text by embedding readLines within
  a for/while loop, but this may be slower than reading in large chunks
  at a time.
\item
  Reading pieces of the file at a time will require the use of a file
  connection in R. For example, the following code could be used to read
  the first few lines of the English Twitter
\end{enumerate}

\begin{itemize}
\tightlist
\item
  dataset:con \textless- file(``en\_US.twitter.txt'', ``r'')
\end{itemize}

readLines(con, 1) \#\# Read the first line of text

readLines(con, 1) \#\# Read the next line of text

readLines(con, 5) \#\# Read in the next 5 lines of text

close(con) \#\# It's important to close the connection when you are
done.

See the connections help page for more information.

\textbf{\emph{Sampling}}.

To reiterate, to build models you don't need to load in and use all of
the data. Often relatively few randomly selected rows or chunks need to
be included to get an accurate approximation to results that would be
obtained using all the data. Remember your inference class and how a
representative sample can be used to infer facts about a population.

\begin{enumerate}
\def\labelenumi{\alph{enumi}.}
\tightlist
\item
  You might want to create a separate sub-sample dataset by reading in a
  random subset of the original data and writing it out to a separate
  file.
\item
  That way, you can store the sample and not have to recreate it every
  time.
\item
  You can use the rbinom function to ``flip a biased coin'' to determine
  whether you sample a line of text or not.
\end{enumerate}

\hypertarget{quiz-1-getting-started}{%
\subsection{Quiz 1: Getting Started}\label{quiz-1-getting-started}}

\textbf{\emph{1. The \verb| en_US.blogs.txt | en\_US.blogs.txt file is
how many megabytes?}}

\begin{Shaded}
\begin{Highlighting}[]
\NormalTok{givendata<-(}\KeywordTok{file.info}\NormalTok{(}\StringTok{"C:/Users/nirma/Documents/Coursera/Data Science/Data Science Capstone/final/en_US/en_US.blogs.txt"}\NormalTok{)}\OperatorTok{$}\NormalTok{size)}\OperatorTok{/}\DecValTok{1024}\OperatorTok{/}\DecValTok{1024}
\KeywordTok{sprintf}\NormalTok{(}\StringTok{"The en_US.blogs.txt file is: %s Megabytes"}\NormalTok{, givendata)}
\end{Highlighting}
\end{Shaded}

\begin{verbatim}
## [1] "The en_US.blogs.txt file is: 200.424207687378 Megabytes"
\end{verbatim}

\textbf{\emph{2. The en\_US.twitter.txt\textbar en\_US.twitter.txt has
how many lines of text?}}

\begin{Shaded}
\begin{Highlighting}[]
\NormalTok{twitterdata<-}\KeywordTok{file}\NormalTok{(}\StringTok{"C:/Users/nirma/Documents/Coursera/Data Science/Data Science Capstone/final/en_US/en_US.twitter.txt"}\NormalTok{, }\DataTypeTok{open=}\StringTok{"rb"}\NormalTok{)}
\NormalTok{total_lines<-}\StringTok{ }\NormalTok{0L }
\ControlFlowTok{while}\NormalTok{(}\KeywordTok{length}\NormalTok{(chunk<-}\KeywordTok{readBin}\NormalTok{(twitterdata,}\StringTok{"raw"}\NormalTok{, }\DecValTok{65536}\NormalTok{))}\OperatorTok{>}\DecValTok{0}\NormalTok{)\{}
\NormalTok{  total_lines<-total_lines}\OperatorTok{+}\KeywordTok{sum}\NormalTok{(chunk}\OperatorTok{==}\KeywordTok{as.raw}\NormalTok{(10L))}
\NormalTok{\}}
\KeywordTok{close}\NormalTok{(twitterdata)}
\KeywordTok{sprintf}\NormalTok{(}\StringTok{"The en_US.twitter.txt file has: %s Lines"}\NormalTok{, total_lines)}
\end{Highlighting}
\end{Shaded}

\begin{verbatim}
## [1] "The en_US.twitter.txt file has: 2360148 Lines"
\end{verbatim}

\textbf{\emph{3. What is the length of the longest line seen in any of
the three en\_US data sets? }}

\begin{Shaded}
\begin{Highlighting}[]
\NormalTok{blogsdata<-}\KeywordTok{file}\NormalTok{(}\StringTok{"C:/Users/nirma/Documents/Coursera/Data Science/Data Science Capstone/final/en_US/en_US.blogs.txt"}\NormalTok{, }\DataTypeTok{open=}\StringTok{"rb"}\NormalTok{)}
\NormalTok{twitterdata<-}\KeywordTok{file}\NormalTok{(}\StringTok{"C:/Users/nirma/Documents/Coursera/Data Science/Data Science Capstone/final/en_US/en_US.twitter.txt"}\NormalTok{, }\DataTypeTok{open=}\StringTok{"rb"}\NormalTok{)}
\NormalTok{newsdata<-}\KeywordTok{file}\NormalTok{(}\StringTok{"C:/Users/nirma/Documents/Coursera/Data Science/Data Science Capstone/final/en_US/en_US.news.txt"}\NormalTok{, }\DataTypeTok{open=}\StringTok{"rb"}\NormalTok{)}
\CommentTok{#Reading Lines}
\NormalTok{blogsdata_lines<-}\KeywordTok{readLines}\NormalTok{(blogsdata)}
\KeywordTok{close}\NormalTok{(blogsdata)}
\NormalTok{blogsdata_L<-}\KeywordTok{summary}\NormalTok{(}\KeywordTok{nchar}\NormalTok{(blogsdata_lines))[}\DecValTok{6}\NormalTok{]}
\NormalTok{blogsdata_L}
\end{Highlighting}
\end{Shaded}

\begin{verbatim}
##  Max. 
## 40835
\end{verbatim}

\begin{Shaded}
\begin{Highlighting}[]
\NormalTok{twitterdata_lines<-}\KeywordTok{readLines}\NormalTok{(twitterdata)}
\end{Highlighting}
\end{Shaded}

\begin{verbatim}
## Warning in readLines(twitterdata): line 167155 appears to contain an embedded
## nul
\end{verbatim}

\begin{verbatim}
## Warning in readLines(twitterdata): line 268547 appears to contain an embedded
## nul
\end{verbatim}

\begin{verbatim}
## Warning in readLines(twitterdata): line 1274086 appears to contain an embedded
## nul
\end{verbatim}

\begin{verbatim}
## Warning in readLines(twitterdata): line 1759032 appears to contain an embedded
## nul
\end{verbatim}

\begin{Shaded}
\begin{Highlighting}[]
\KeywordTok{close}\NormalTok{(twitterdata)}
\NormalTok{twitterdata_L<-}\KeywordTok{summary}\NormalTok{(}\KeywordTok{nchar}\NormalTok{(twitterdata_lines))[}\DecValTok{6}\NormalTok{]}
\NormalTok{twitterdata_L}
\end{Highlighting}
\end{Shaded}

\begin{verbatim}
## Max. 
##  213
\end{verbatim}

\begin{Shaded}
\begin{Highlighting}[]
\NormalTok{newsdata_lines<-}\KeywordTok{readLines}\NormalTok{(newsdata)}
\KeywordTok{close}\NormalTok{(newsdata)}
\NormalTok{newsdata_L<-}\KeywordTok{summary}\NormalTok{(}\KeywordTok{nchar}\NormalTok{(newsdata_lines))[}\DecValTok{6}\NormalTok{]}
\NormalTok{newsdata_L}
\end{Highlighting}
\end{Shaded}

\begin{verbatim}
##  Max. 
## 11384
\end{verbatim}

\textbf{\emph{4.In the en\_US twitter data set, if you divide the number
of lines where the word ``love'' (all lowercase) occurs by the number of
lines the word ``hate'' (all lowercase) occurs, about what do you get?}}

\begin{Shaded}
\begin{Highlighting}[]
\NormalTok{lovehate<-}\KeywordTok{length}\NormalTok{(}\KeywordTok{grep}\NormalTok{(}\StringTok{"love"}\NormalTok{, twitterdata_lines))}\OperatorTok{/}\KeywordTok{length}\NormalTok{(}\KeywordTok{grep}\NormalTok{(}\StringTok{"hate"}\NormalTok{, twitterdata_lines))}
\KeywordTok{sprintf}\NormalTok{(}\StringTok{"We get around: %s"}\NormalTok{, lovehate)}
\end{Highlighting}
\end{Shaded}

\begin{verbatim}
## [1] "We get around: 4.10859156202006"
\end{verbatim}

\textbf{\emph{5. The one tweet in the en\_US twitter data set that
matches the word ``biostats'' says what?}}

\begin{Shaded}
\begin{Highlighting}[]
\NormalTok{biostat_tweet<-}\KeywordTok{grep}\NormalTok{(}\StringTok{"biostats"}\NormalTok{, twitterdata_lines, }\DataTypeTok{value=}\OtherTok{TRUE}\NormalTok{)}
\NormalTok{biostat_tweet}
\end{Highlighting}
\end{Shaded}

\begin{verbatim}
## [1] "i know how you feel.. i have biostats on tuesday and i have yet to study =/"
\end{verbatim}

\textbf{\emph{6. How many tweets have the exact characters ``A computer
once beat me at chess, but it was no match for me at kickboxing''. (I.e.
the line matches those characters exactly.)}}

\begin{Shaded}
\begin{Highlighting}[]
\NormalTok{tweet_match<-}\KeywordTok{grep}\NormalTok{(}\StringTok{"A computer once beat me at chess, but it was no match for me at kickboxing"}\NormalTok{, twitterdata_lines)}
\NormalTok{tweet_match}
\end{Highlighting}
\end{Shaded}

\begin{verbatim}
## [1]  519059  835824 2283423
\end{verbatim}

\end{document}
